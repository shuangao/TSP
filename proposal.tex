\documentclass[12pt,onecolumn]{article}
\usepackage{latexsym}
\usepackage{url}
\usepackage{float}
\usepackage{amsmath}
\usepackage{geometry}
\usepackage{subfig}
\geometry{left=2.0cm,right=2.0cm, top=2.0cm, bottom=2.0cm}
\usepackage{fancyhdr}
\usepackage{longtable}
\usepackage{url}
\usepackage{leftidx}
\usepackage{graphicx}
\usepackage{epstopdf}
\usepackage{subfig}
\usepackage{tikz}
\usepackage{ifthen}
%\usepackage{subcaption}
\usepackage{caption}
\newcommand\figref[1]{Figure \ref{#1}}
\linespread{1.5}
\author{Shuang Ao \\
Supervisor: Professor Charles Ling
}
\title{
\large\textbf{Unsupervised Feature Learning For Image Recognition}
}
\frenchspacing
\begin{document}
\maketitle
\tableofcontents
\newpage
\begin{abstract}
In image recognition, the success of machine learning algorithm is mainly depends on the image representation. Different representations can lead to different views of the image data. Although, the specific domain knowledge can help us for representation, sophisticate method for image representation is still crucial. In this TSP, firstly the background knowledge is introduced, including some basic feature extraction methods. Deep learning, which is introduced in Section 2, is the most popular method for feature extraction technique which has gained great reputations in both academic and industrial area. Finally, I present a novel method that is inspired by the idea of deep learning. In this method, I try to use patch combination as well as some ranking methods for image representation. The defects of this method can be my future work.

\end{abstract}

\section{Introduction to Image Filter}\label{sec:intro}
Image recognition is an important part in Computer Vision which includes many advanced machine learning techniques. A pattern recognition system (PRS) is an automatic system that aims at classifying the input pattern into a specific class\cite{kpalma2007overview}. An image recognition system
always proceeds in two steps: (1) generate several important features (descriptors or a book of code) (2) use a classifier
to distinguish the patterns with all these features. Compared to our human learning procedures, some prior knowledge should acquired
before the recognition which turns the learning procedure into be a supervised learning.

There are four major methodologies in image recognition:
\begin{itemize}
  \item Statistical approach. Here features are converted into a vector to present the pattern which can be a very easy to handle. This method
        is wildly and intensively used in practice.

  \item Syntactic approach. This method analysis the relationship between features and patterns are described in a hierarchical structure. For example
        (see \figref{sys}), the image on the right hand can be presented as the string ${d b a b c b a b d b a b c b a b}$.The system parses the set of extracted features using a kind of predefined grammar. If the whole features extracted from a pattern can be parsed to the grammar then the system has recognised the pattern. Unfortunately, grammar-based syntactic pattern recognition is generally very difficult to handle.

  \item Template matching. This can be the most intuitive way to localize and identify shapes in an image. Here, the method looks for the parts of the
        image which match the template. For each possible position in the image, it compares with each possible rotation or each possible geometric transformation of the template. According to certain metrics that can measure the similarity of these, the approach returns the most similar one. Since it has to compute all kinds of possibilities, this could be a very computational expensive approach.
  \item Neural networks. The an artificial neural network (ANN) had been used extensively before 2006 for image recognition. In 2006,  G. Hinton introduced "deep learning", a new method that can learning image features with fast unsupervised procedure which led to a revolutionary in image recognition\cite{hinton06}. We will
      discuss this method in details in section \ref{dl}.
\end{itemize}
\begin{figure}
  \centering
  \includegraphics [width=6in] {fig/sys.png}\\
  \caption{Example of syntactic description features}\label{sys}
\end{figure}
There are some simple dependencies when representing the image patterns. One reason why explicitly dealing with image patterns is interesting is because they can be convenient to express many general priors about the world around us, i.e., priors that are not task specific but would be likely to be useful for a learning machine to solve AI tasks\cite{bengio2007scaling}. These general priors can sometime be used to help us design more powerful learners to discover the underlying factors that the data may reveal. Here are some examples of the general-purpose priors:
\begin{itemize}
  \item Smoothness: Assumes that the function $f$ to be learning for evaluation must satisfy that for any two patterns $x,y$, if $x\approx y, f(x)\approx f(y)$.
  %\item Multiple explanatory factors:
  \item A hierarchical organization of explanatory factors: Some patterns are more abstract than the others. These abstract patterns can be defined in terms of other patterns hierarchically.
  %\item Semi-supervised learning:
  %\item Shared factors across tasks
  %\item Manifolds:
  \item Natural clustering: Different values of categorical variables such as object classes are associated with separate manifolds. Humans have named categories and classes because of such statistical structure (discovered by their brain and propagated by their culture), and machine learning tasks often involve predicting such categorical variables.
 % \item Temporal and spatial coherence
  \item Sparsity: For any given observation $x$, only a small fraction of the possible factors are relevant\cite{olshausen1996emergence}. This means just a small fraction of the patterns can represent a image well.
  \item Simplicity of factor dependencies: The good abstract patterns can be related to each other through simple, typically linear combination.This can be seen in many neural networks when the top layer is just linear combination of the hidden units.
\end{itemize}







    \subsection{Kernel Based Method for Image Recognition}
\input{kbased.tex}
\newpage
\section{Deep Learning Feature Representation}\label{dl}
     \begin{figure}[h]
  \centering
  \includegraphics[scale = .5]{fig/dbn.png}\\
  \caption{Deep learning procedure}
\end{figure}
In 2006, a new feature learning and deep learning method was initiated by G.Hinton and has brought great impact in both academic and industrial field. This deep learning method is a kind of deep neural network which contains several layers. The core idea, referred to as \emph{greedy layerwise unsupervised pretraining} was to learn the bottom to top hierarchical features in a deep neural network. In deep learning, each layer is trained separately which makes is really fast and the output of the lower layer is the input of the higher one. When using greedy layerwise unsupervised pretraining, the two conjunctival layers can regarded as a Restrict Boltzman Machine (RBM) and the \emph{Energy} of the RBM is defined as\cite{Hinton12}:
\begin{equation}\label{eq:engery}
  E(v,h) =  - \sum\limits_{i \in visible} {{a_i}{v_i}}  - \sum\limits_{j \in hidden} {{b_j}{h_j}}  - \sum\limits_{i,j} {{v_i}{h_i}{w_{ij}}}
\end{equation}
Here, $v_i$, $h_j$ are the visible value (input) and the hidden value (output) respectively, and $a_i$, $b_j$ are their bias . $w_{ij}$ is the weight of the visible node $i$ and hidden node $j$. The probability that the RBM can recall the visible vector, $v$, is defined as:
\begin{equation}
\begin{gathered}
  p(v) = \frac{1}{Z}\sum\limits_h {{e^{ - E(v,h)}}} \\
   Z = \sum\limits_{i,h} {{e^{ - E(v,h)}}}  \\
\end{gathered}
\end{equation}
The derivative of the log probability of a training vector with respect to a weight is:
\begin{equation}
\begin{gathered}
  \frac{{\delta \log p(v)}}{{\delta {w_{ij}}}} = {\left\langle {{v_i}{h_j}} \right\rangle _{data}} - {\left\langle {{v_i}{h_j}} \right\rangle _{model}}\\
  {\left\langle {{v_i}{h_j}} \right\rangle _{model}} \approx {\left\langle {{v_i}{h_j}} \right\rangle _{recon}}\\
  \end{gathered}
\end{equation}\label{eq:weight}

\begin{figure}[h]
\centering
\def\layersep{1.5cm} % Gap between visible & hidden units
\def\numvis{3} % Number if visible units
\def\numhid{4} % Number of hidden units
\begin{tikzpicture}[
    node distance=\layersep,
    line/.style={<->,shorten >=2pt,shorten <=2pt,>=stealth}
    ]
    \tikzstyle{neuron}=[circle,fill=black!25,minimum size=21pt,inner sep=0pt];
    \tikzstyle{visible neuron}=[neuron, fill=green!50];
    \tikzstyle{hidden neuron}=[neuron, fill=orange!75];
    \tikzstyle{annot}=[text width=4em];

    % Iterate over visible units
    \foreach \name / \y in {1,...,\numvis}
        \node[visible neuron] (V\name) at (\y,0) {$v_\y$};

    % Iterate over hidden units
    \foreach \name / \y in {1,...,\numhid}
        % Calculate the appropriate offset for the hidden unit based on the
        % number of visible units.
        \pgfmathparse{\y + (\numvis - \numhid) * 0.5}
        \node[hidden neuron] (H\name) at (\pgfmathresult, \layersep) {$h_\y$};

    % Draw a double-arrow connector between the each unit
    \foreach \source in {1,...,\numvis}
        \foreach \dest in {1,...,\numhid}
            \draw[line] (V\source) -- (H\dest);

    % use if/then/else to decide where to draw the labels
    \ifthenelse{\numvis > \numhid}
    {
        \node[annot,left of=V1, node distance=1cm] (hl) {Visible};
        \node[annot,above of=hl] {Hidden};
    }
    {
        \node[annot,left of=H1, node distance=1cm] (hl) {Hidden};
        \node[annot,below of=hl] {Visible};
    }
\end{tikzpicture}
\label{fig:rbmtopology}
\caption[RBM Topology]{Restricted Boltzmann topology with \numvis{} visible units and \numhid{} hidden units.}
\end{figure}

Even though,  the ${\left\langle {{v_i}{h_j}} \right\rangle _{model}}$ can't be calculated, Hinton proposed a fast learning procedure, called \emph{Contrastive Divergence}, that can estimate it by ignoring the tricky term in the objective function and proved that this can lead to the convergence. Sutskever and Tieleman have shown that this procedure is not following the gradient of any function\cite{SutskeverT10}.
\begin{figure}
  \centering
  % Requires \usepackage{graphicx}
  \includegraphics[scale=.5]{fig/deeplearning.pdf}\\
  \caption{Hierarchical Representation of Deep Learning}
\end{figure}

After this greedy layerwised unsupervised pretraining, the configuration of the deep network can be the initial of the supervised deep neural network predictor\cite{Srivastava13}. And the stochastic gradient descent can be applied for further optimization. The initial configuration after greedy layerwised unsupervised pretraining can make this deep network get rid of the local minimum since the supervised procedure is not based on any gradient method.

It is interesting to ask why this greedy layerwise unsupervised pretraining can help in supervised learning\cite{ErhanCBV10}. The curriculum greedy layerwise unsupervised pretraining tries to build some intermediate representations rather than some distinguishable representations. This is nicely related to curriculum learning idea which indicates that it is much more easier to learn some simpler concept first and build a much more complicated ones on top of the simpler ones\cite{BengioICML09}. For a deep supervised classifier, in the unsupervised training procedure, the deep neural network is the stacked RBMs which can initialize in a \emph{good} local minimum where the \emph{good} refers to low generalization error\cite{Bengio13}.

Stacked RBMs has been developed for many large learning problems as the feature extractor for other supervised classifiers. In these cases, stacked RBM is used as both feature extractor to learn the hierarchical representations and autoencoder to evaluate the similarity between the features and the images.  However, single-layer RBM can still be a good predictor for a supervised learning problems. H.Larochelle used the RBM stand-alone to build the discriminative RBM on character recognition without replying on any other classifier\cite{LarochelleB08}. Even though, for the large learning task, deep neural network with millions or even billions of nodes can get some good results\cite{krizhevsky2012imagenet}\cite{LeRMDCCDN12}, in some simple shadow network can still do well. It's a big waste to use a cannon to kill a mosquito.

When training the deep neural network, there are many details that can speed the convergency of the stacked RBMs to achieve a better local minimum. Cho et al. proposed a adaptive learning rate for the RBM training as well as a novel gradient estimator that takes into account the invariance of the model to flipping hidden unit bits and inverting signs of corresponding weight vectors\cite{cho2011}.  For the choice of the hidden units, several experiments have shown that nonlinearity could influence both training and generalization performance\cite{GlorotB10}\cite{GlorotBB11}\cite{NairH10}. Moreover, the initial weights can affect the training results. A good initialization can substantially reduce the difficulty of training the deep neural network. With a sparse initialization or using often saturated nonlinearities, it is more promising to make the hidden units specialized\cite{martens2010deep}. For the training method, there is still a debate between online learning method such as stochastic gradient descent and using large mini batch(thousands of samples)\cite{LeNCLPN11},  even though, the stochastic gradient descent and its invariants are more popular and have won many competitions. As we can see in Equation (\ref{eq:weight}), most of the computation in updating the weights can be done with matrix operations, it is more convenient to use GPU to do these jobs which is designed to accelerate the matrix operations\cite{raina2009large}. Normally, training the huge deep network on GPU can be at least 10 times faster than training on CPU.

Despite great achievement in both academic and industrial area, there are still some criticisms, one of which is that there are so many hyperparameters and variants that exploring the configuration and the architecture is really an art\cite{LeCunBOM12}. To solve this problem, many automating hyperparameters searching works have been proposed recently which makes it more convenient and efficient\cite{BergstraB12}\cite{BergstraBBK11}\cite{snoek2012practical}. To deal with large quantities of image recognition problem, deep learning can be the first choice. As a neural network, it can easily get rid of the local  minimum, which is the biggest issue in all the large neural network training procedure, with just a little prior knowledge through the efficient unsupervised training. However, recent work on large quantities of labeled data shows that with a proper initialization, the pure deep supervised network can still be successfully trained without any layerwised pre-training\cite{cirecsan2010deep}\cite{glorot2011deep}\cite{seide2011conversational}. And in this situation, the layerwised unsupervised pretraining can bring little or even no improvement over the supervised one when training for long enough.

In addition, deep learning will be the most promising method for image recognition in the next few years. Even though pure supervised learning can compete with deep learning while there are enough labeled data for training, still deep learning and unsupervised pre-training for image recognition would be the future. As in real world, we can always obtain enough data while just a very small fraction can be labeled. Unsupervised feature extraction techniques can be a more realistic method. 


 \newpage
\section{K-means Feature Representation}\label{km}
    \input{kmeans.tex}
\newpage
\section{Conclusion}
 This TSP has mainly cover three parts of work:
\begin{itemize}
  \item In Section \ref{sec:intro}, some concepts and the kernel based image filtering technique are introduced. The kernel based image process method is the most important technique in image recognition which is widely used in feature extraction. Kernels are often used in convolutional neural network which can be implemented very fast with the help of GPU. Therefore, the large convolutional neural network is the first choice for large real world image learning task. 
  \item In Section \ref{dl}, the deep learning method is introduced. Deep learning is the state-of-the-art and the most popular model for image recognition. It shows great power on both learning the features and image representation. Deep learning uses the stacked RBM to learning the hierarchical feature representation and use the Contrastive Divergence strategy to training the model in an unsupervised way. This feature learning idea has given inspiration to many other unsupervised feature extract methods as well as the method proposed in Section \ref{km}.
  \item In the Section \ref{km}, the kmeans based feature extract method is proposed. Inspired by the unsupervised feature extraction idea in Deep learning, I try to use kmeans to find the templates as the features for image representation. The patch combination can be the novel part of this method which combines the centroids in different location to generate the "super patch". After patch combination, the unsupervised and supervised patch ranking methods are used to eliminate the sparse, redundant and irrelevant centriods. Finally the encoding methods is introduced.
\end{itemize}

In the kmeans based feature representation method, there are less hyperparemeters so that the background knowledge becomes less important. Still there are many challenges for this method: 
\begin{itemize}
  \item In patch ranking, even though, the bad centroids and irrelevant ones are ranked low, it is still unknow for us to know what is the optimal number of features get for learning the specific problem without the help of learning curve. Thus there are a lot of work to determine the optimal number of features for this method. In my future work, some score may be used to evaluate the centroids.
  \item There is still no good solution for the encoding problem as its complexity. There are too many variables that need to be considered and still many work for me to find the best metric for the similarity measurement.
\end{itemize}
In the next few terms, all these problems above should be the main work for me and hopefully, with the supervision of Dr. Charles Ling, I can find some solutions for them and improve this method. 
 %\\
In K-means algorithm, the performance of the algorithm is mainly determined by the parameter K, which denotes the number of cluster. However, in most of the situation, it is not possible to determine a proper K. For object recognition, it could be a challenge for human to determine K for the algorithm, even though, human can distinguish objects by some features, such as color, shape, texture, etc. Many works have been done to heuristically determine the number of cluster automatically. Ray et.al \cite{ray1999determination} proposed a method that starts K from 2 to $K_{max}$, which is defined prior as the upper limit for K. Figueiredo et.al. \cite{figueiredo2002unsupervised} estimated K with minimum message length (MML) criteria and Gaussian mixture model. These methods try to learn the data distribution naturally without much human involvement, so it makes the algorithms complicate and slow. 
  \begin{figure}[ht]
    \subfloat[Original Image]{%
      \includegraphics[width=0.45\textwidth]{fig/multdim/orig.png}
    }
    \hfill
    \subfloat[K=5]{%
       \includegraphics[width=0.45\textwidth]{fig/multdim/k5.png}
    }\\
    \subfloat[K=6]{%
      \includegraphics[width=0.45\textwidth]{fig/multdim/k6.png}
    }
    \hfill
    \subfloat[K=10]{%
       \includegraphics[width=0.45\textwidth]{fig/multdim/k10.png}
    }
    \caption{Clustering for image segmentation}
    \label{fig:cluster seg}
  \end{figure}
  
Normally smaller data can be better clustered while large volume data or data with high dimension could be a hard task for any clustering algorithm. In object recognition, our human classify objects mainly by their shapes and colors. Human have many prior knowledge that can teach the algorithm to use and this could make it simple and fast. Suppose we have to group the two-dimensional dataset in \figref{fig:multidim}.

\begin{figure}[ht]
  \centering
  % Requires \usepackage{graphicx}
  \includegraphics[width=0.7\textwidth]{fig/multdim/example.png}\\
  \caption{How multi-dimensional clustering works}
  \label{fig:multidim}
\end{figure}

 %  
  \begin{figure}[ht]
    \subfloat[Original Image]{%
      \includegraphics[width=0.45\textwidth]{fig/imagextend/orig.png}
    }
    \hfill
    \subfloat[Distance Map]{%
       \includegraphics[width=0.45\textwidth]{fig/imagextend/distmap.png}
    }\\
    \subfloat[0.8]{%
      \includegraphics[width=0.45\textwidth]{fig/imagextend/8thre.png}
    }
    \hfill
    \subfloat[1.0]{%
       \includegraphics[width=0.45\textwidth]{fig/imagextend/1thre.png}
    }  \\
     \subfloat[1.5 ]{%
      \includegraphics[width=0.45\textwidth]{fig/imagextend/15thre.png}
    }
    \hfill
    \subfloat[2]{%
       \includegraphics[width=0.45\textwidth]{fig/imagextend/2thre.png}
    }
    \caption{Clustering for image segmentation}
    \label{fig:extendthr}
  \end{figure}
 %\begin{table}
  \centering
  \begin{tabular}{|c|c|c|c|c|c|c|}
     \hline
     % after \\: \hline or \cline{col1-col2} \cline{col3-col4} ...
     method &   Time   &   Scale    &  Rotation   & Blur    &   Illumination   &   Affine\\
     \hline
    Sift   &   common  &  best    &   best     &  common  &   common   &    good\\
    \hline
    PCA-sift  &  good  &    good   &   good   &   best   &   good    &    best\\
    \hline
    Surf    &  best   &    common  &  common   &  good   &    best     &    good\\
     \hline
   \end{tabular}
  \caption{XX}
\end{table}


\newpage

\bibliographystyle{abbrv}
\bibliography{research}

%\newpage
%\appendix
%\section*{Appendix}
%\subsection{cool}

%\renewcommand{\appendixname}{Appendix~\Alph{section}}

   % \section{Some Patch Combination Results}


\end{document}
